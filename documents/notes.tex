\documentclass{article}

\usepackage{amsmath, amsthm, amssymb, amsfonts}
\usepackage{thmtools}
\usepackage{graphicx}
\usepackage{setspace}
\usepackage{geometry}
\usepackage{float}
\usepackage{hyperref}
\usepackage[utf8]{inputenc}
\usepackage[english]{babel}
\usepackage{framed}
\usepackage[dvipsnames]{xcolor}
\usepackage{tcolorbox}

\colorlet{LightGray}{White!90!Periwinkle}
\colorlet{LightOrange}{Orange!15}
\colorlet{LightGreen}{Green!15}

\newcommand{\HRule}[1]{\rule{\linewidth}{#1}}


\setstretch{1.2}
\geometry{
    textheight=9in,
    textwidth=5.5in,
    top=1in,
    headheight=12pt,
    headsep=25pt,
    footskip=30pt
}

% ------------------------------------------------------------------------------

\begin{document}

% ------------------------------------------------------------------------------
% Cover Page and ToC
% ------------------------------------------------------------------------------

\title{ \normalsize \textsc{}
		\\ [2.0cm]
		\HRule{1.5pt} \\
		\LARGE \textbf{\uppercase{Documento di Note}
		\HRule{2.0pt} \\ [0.6cm] \LARGE{In questo documento vengono riportati i concetti affrontati durante lo stage presso Synclab S.r.L.} \vspace*{10\baselineskip}}
		}
\date{}
\author{\textbf{Author} \\ 
		Marco Brugin \\
		Synclab S.r.L. \\
		\today}

\maketitle
\newpage

\tableofcontents
\newpage

\section{Streaming ad eventi}
È una pratica di acquisizione dei dati in tempo reale da fonti di eventi come database, flussi di eventi; memorizando tutto ciò per un recupero futuro di tali informazioni, reagendo a flussi di eventi in tempo reale. Inoltre garantisce un flusso  continuo di info corrette nel posto giusto e  al momento giusto.
\subsection{Utilizzi}
\begin{itemize}
    \item per transazioni
    \item per servizzi IOT di vario genere
    \item monitoraggio sanitario 
    \item ovunque ci sia la necessità di trattare grandi moli di dati efficientemente
\end{itemize}
\section{Apache \textbf{Kafka}}
\textbf{Kafka} è una piattaforma open source che combina 3 funzionalità in modo da poter soddisfare i casi d'uso sopra citati:
\begin{itemize}
    \item pubblica e sottoscrive flussi di eventi, importandoli ed 	esportandoli da altri sistemi;
    \item archivia tali flussi in modo affidabile e duraturo;	
    \item elabora flussi di eventi in real time o in modo retrospettivo.
\end{itemize}
\subsection{Utilizzo}
\textbf{Kafka} opera su una architettura distribuita. Può essere distribuito e utilizzato in vari modi tra cui virtual machine e container, on-promise, o servizi cloud.

\subsection{Funzionamento}
\textbf{Kafka} nasce come sistema distribuito che opera su nodi che comunicano tramite protocollo \textbf{TCP} ad alte prestazioni. Data la sua natura distribuita implementa capacità di fault tollerance con rimpiazzo dei nodi guasti. 
\textbf{Kafka} è costituito da due componenti essenziali: server e client.
\subsubsection{Server}
\textbf{Kafka} viene eseguito come un cluster di uno o più server. Alcuni fanno da \textbf{broker}: livello di archiviazione. Altri assolvono il compito di \textbf{\textbf{Kafka} Connect}: importano e esportano i dati sotto forma di  flussi di eventi che permette di interagire con altri sistemi esistenti.
\subsubsection{Client}
Consentono di scrivere applicazioni distribuite e microservizi che leggono, scrivono ed elaborano flussi di eventi in parallelo, su larga scala e con fault tollerance anche in caso di problemi di rete o guasti della macchina.
Esitono molti client per diversi linguaggi di programmazione.
\subsection{Garanzie di funzionamento}
In \textbf{Kafka} esistono produttori e consumatori che producono e sottoscrivono eventi. 
Gli uni sono sono indipendenti l'uno dall'altro, ciò permette di raggiungere alcune delle seguenti garanzie:
\begin{itemize}
    \item \textbf{Al massimo una volta}: i messaggipossono andare persi ma mai riconsegnati;
    \item \textbf{Almeno una volta}: i mesaggi non 		vengono mai persi ma possono essere 		riconsegnati;
    \item \textbf{Una solo volta}: i messaggi non  		vanno persi e sono consegnati una 		sola volta; è la garanzia maggiormente desiderabile.
\end{itemize}
\subsection{Gestione degli eventi}
Gli eventi sono organizzati in topic o argomenti. Gli eventi possono essere letti da più consumatori. 
Un evento anche se consumato non viene eliminato, può essere impostato un timeout di mantenimento dell'evento.
I topic sono partizionati su più nodi. Tale partizionamento consente ai client di leggere e scrivere dati da/a molti broker. Quando un nuovo evento viene emesso questo si aggiunge all rispettiva partizione. Sarà \textbf{\textbf{Kafka}} a garantire che gli eventi vengano letti nell'ordine in cui sono stati scritti.
\subsection{Interfacce presentì}
\subsubsection{\textbf{Kafka} Producer}
L’interfaccia Producer permette alle applicazioni di inviare i flussi di dati ai broker di un cluster di Apache per categorizzarli e salvarli (nei topic già citati).
\subsubsection{\textbf{Kafka} Consumer}
L’interfaccia Consumer consente ai consumatori di Apache \textbf{Kafka} di  ricevere l’accesso ai dati, salvati nei topic di un cluster.
 \subsubsection{\textbf{Kafka} Connect}
    L’interfaccia Connect  consente di impostare produttori e consumatori riutilizzabili che collegano i topic di \textbf{Kafka} con le applicazioni o le banche dati esistenti.
\subsection{Casi d'uso}
\subsubsection{Notifica dell'evento}
\textbf{Kafka} permette di trasmette semplicemente eventi per avvisare altri sistemi di un cambiamento nel suo dominio.
\subsubsection{Trasferimento di stato portato da eventi}
In questo utilizzo il destinatario dell'evento ottiene anche i dati di cui ha bisogno per eseguire azioni aggiuntive sui dati richiesti.
\subsubsection{Approvvigionamento di eventi}
In questo caso il sistema consente di descrivere ogni cambiamento di stato in un sistema come un evento, con ogni evento registrato in sequenza cronologica. Di conseguenza, il flusso di eventi stesso diventa la principale fonte di verità del sistema.
\subsection{Applicazioni}
\subsubsection{Messaggistica}
I broker di messaggi vengono utilizzati per disaccoppiare l'elaborazione del messaggio dal produttore dello stesso.
\textbf{Kafka} rispetto ai sistemi di messaggistica tradizionale ha una migliore velocità, partizionamento integrato e tolleranza agli errori.
L'utilizzo è basso, ma in ambiti in cui si richiede una bassa latenza, le garanzie fornite da \textbf{Kafka} sono alla pari dei tradizionali sistemi di messagistica, che lo rendono lo rendono una buona soluzione per applicazioni di elaborazione di messaggi su larga scala.

\subsubsection{Monitoraggio di siti web}
È il caso d'uso d'origine di \textbf{Kafka} , ha la caratteristica di generare un elevato volume di messaggi. Lo scopo era quello di ricostruire le attività degli utenti come insieme di eventi genrati dalle azioni dell'utente.
\subsubsection{Metrica}
Indica l'aggregazione di dati di monitoraggio provenienti da varie fonti per eseguire statistiche.
\subsubsection{Aggregazione di registri}
\textbf{Kafka} ha anche la capacità di estrarre i dati dai file di log fisici e fornisce una astrazione di tali sotto forma di flusso di messaggi. Ciò permette di garantire una più bassa latenza di elaborazione e supporto per più origini dati.
\subsubsection{Elaborazione del flusso}
È possibile andare a creare una  data-pipeline in cui i dati grezzi vengono consumati dagli argomenti di \textbf{Kafka}, aggregati, trasformati fino ad ottenere unn dato elaborato.
\subsubsection{Approvvigionamento di eventi}
È possibile utilizzare  di \textbf{Kafka} in aplicazioni cui cambiamenti di stato vengono registrati come sequenze di record in ordine temporale. \textbf{Kafka} permette il supporto per dati di registro memorizzati molto grandi tanto che lo rende un ottimo back-end di tali applicazioni. 
\subsubsection{Registro commit}
\textbf{Kafka} può fungere da sorta di log di commit esterno per un sistema distribuito. Il registro aiuta a replicare i dati tra i nodi e funge da meccanismo di risincronizzazione per consentire ai nodi non allineati di ripristinare i propri dati.

\subsection{Apache Kafka VS EDA}
L'emissione di un evento indica che qualcosa è accaduto e può essere visto come un agglomerato di dati atomico in grado di soddisfare l'evento stesso. 
\textbf{Kafka} è un sistema di streaming di eventi che gestisce un flusso continuo di eventi. Inoltre Kafka memorizza tali in modo duraturo per il successivo recupero, analisi o elaborazione in tempo reale e li inoltri a varie destinazioni secondo necessità.
\\
D'altra parte in una \textbf{EDA} (Event Driven Architecture) viene generati degli eventi che un agente acquisisce e risponde a tale evento.
per utilizzare \textbf{Kafka} in un sistema \textbf{EDA} la chiave è andare a sfruttare il disaccopiamento: invece di effetuare un polling continuo di verifica della presenza di nuovi dati, basterà ascoltare il verificarsi di un evento per agire. Inoltre grazie all'approcio sviluppato da \textbf{Kafka} un evento una volta soddisfatto non viene eliminato, ma conservato per un periodo ti tempo predeterminato, pertanto un evento potrà essere letto da più consumatori e potrà essere utilizzato per soddisfare una varietà di richieste.


\bibliographystyle{IEEEtran}
\bibliography{https://Kafka.apache.org/documentation/}
\bibliography{https://hevodata.com/learn/kafka-event-driven-architecture/#events}

% ------------------------------------------------------------------------------

\end{document}
